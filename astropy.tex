\documentclass[12pt,preprint]{aastex}

\usepackage{natbib}
\bibpunct{(}{)}{;}{a}{}{,}

\begin{document}

\title{Astropy: A Community Python Package for Astronomy}

% The author list is not final, and the order of authors may still change.
% Additional authors may be added based on non-commit-based contributions to
% the project.

% Please wrap text to 78 characters.

\author{
The Astropy Collaboration,
% Coordination committee
Erik Tollerud,
Thomas P. Robitaille,
Perry Greenfield,
% Developers (other than coordination committee).
Michael Droettboom,
Tom Aldcroft,
Erik Bray,
Matt Davis,
Adam Ginsburg,
Adrian Price-Wheelan,
Wolfgang Kerzendorf,
Alexander Conley,
Neil Crighton,
Kyle Barbary,
Demitri Muna,
Henry Ferguson,
Frederic Grollier,
Pransanth Nair,
Hans M. G\"unther,
Christoph Deil,
Simon Conseil,
Julien Woillez,
Roban Kramer,
James Turner,
Leo Singer,
% Other contributors, alphabetical, including coordination meeting
% participants
Eric Barron,
Azalee Bostroem,
Doug Burke,
Andy Casey,
Steve Crawford,
Nadia Dencheva,
Justin Ely,
Tim Jenness,
Kathleen Labrie,
Pey Lian Lim,
Francesco Pierfederici,
Andrew Pontzen,
Andy Ptak,
Brian Refsdal,
Mathieu Servillat,
Ole Streicher
}

\begin{abstract}
We present the first public version of the open-source and community-developed
Python package, Astropy. This package aims to provide core Astronomy-related
functionality to the community, including for example support for
domain-specific file formats such as Flexible Image Transport System (FITS)
files, Virtual Observatory (VO), and common ASCII table formats, unit and
physical quantity conversions, physical constants specific to Astronomy,
celestial coordinate and time transformations, world coordinate system (WCS)
support, generalized containers for representing gridded as well as tabular
data, and a framework for cosmological transformations and conversions.
Significant functionality is under active development, such as a model fitting
framework, VO client and server tools, and aperture and point spread function
(PSF) photometry tools. Participation in development is open to anyone
interested in joining the project.
\end{abstract}

\tableofcontents


\section{Introduction}

% Authors: Tom R., Erik T., Perry G.

The Python programming language\footnote{\url{http://www.astropy.org}} has
arguably been one of the fastest-growing programming languages in the
Astronomy community in the last decade. While there have been a number of
efforts to develop Python packages for Astronomy-specific functionality, these
efforts have been fragmented, and several dozens of packages have been
developed across the community with little or no coordination, leading to
duplication and a lack of homogeneity across packages. This in turn has made
it difficult for users to get set up with all the required packages needed in
an Astronomer's toolkit. Since a number of these packages depend on individual
or small groups of developers, packages are sometimes no longer maintained, or
simply become unavailable, which is detrimental to long-term research and
research reproducibility.

The Astropy project was started in 2011 out of a desire to bring together
developers across the field of Astronomy in order to coordinate efforts to
develop a common Python library covering much of the Astronomy-specific
functionality needed by researchers, with the aim of complementing more
general standard scientific packages such as NumPy
\citep{oliphant2006guide,van2011numpy} and SciPy \citep{jones2001scipy}, which
are invaluable for numerical array-based calculations, and more general
scientific algorithms (e.g. interpolation, integration, clustering, etc.)
respectively. To date, over 130 people are signed up to the
\textit{development} mailing list for the Astropy project\footnote{
\url{https://groups.google.com/forum/?fromgroups\#!forum/astropy-dev}}.

Most efforts in the Astropy project to date have gone towards developing the
core \texttt{astropy} package. However, we note the scope of the Astropy
project is not simply to create a core package, but more generally to bring
together authors of existing packages to work together and agree on including
a common set of functionality in the core package (and in some cases
deprecating their packages if all functionality ends up in \texttt{astropy}),
since the aim is ultimately to simplify the landscape of packages available to
users. In addition, the Astropy project includes more specialized Python
packages (which we call \textit{affiliated} packages) that are not included in
the core package for various reasons: for some the functionality is in early
stages of development and is not robust; or the license is not compatible with
Astropy; the package includes large files; or the functionality is mature, but
too specific to be included in a core package.

In this paper, we announce the first public release (v0.2) of the
\texttt{astropy} package. We provide an overview of the current capabilities
(\S\ref{sec:capabilities}), our development workflow (\S\ref{sec:workflow}),
and planned functionality (\S\ref{sec:future}).

\section{Capabilities}

\label{sec:capabilities}


\subsection{Units, Quantities, and Physical Constants}

% Authors: Perry, Mike, Adrian, Tom R.


\subsection{Time}

% Authors: Tom A.


\subsection{Celestial Coordinates}

% Authors: Erik T., Adrian, Demitri


\subsection{Tables and Gridded data}

% Authors: Tom A., Tom R., Wolfgang K., Erik T.

The \texttt{astropy.table} and \texttt{astropy.nddata} sub-packages contain
classes (\texttt{Table} and \texttt{NDData}) that allow users to represent
astronomical data in the form of tables or n-dimensional gridded datasets,
including meta-data (for example units).

The \texttt{Table} class provides a high-level wrapper to Numpy structured
arrays, which are essentially arrays that have fields (or columns) with
heterogeneous data types, and any number of rows. Numpy structured arrays are
however difficult to manipulate or modify, so the \texttt{Table} class makes
it easy for users to create a table from columns, add/remove columns or rows,
and mask values from the table. In addition, tables can be easily read/written
from/to common file formats using the \texttt{Table.read} and
\texttt{Table.write} methods. In addition to providing easy manipulation and
input/output of table objects, the \texttt{Table} class allows units to be
specified for each column using the \texttt{astropy.units} framework, and also
allows the \texttt{Table} object to contain arbitrary meta-data (stored in
\texttt{Table.meta}).

The \texttt{NDData} class ...

\subsection{File Formats}


\subsubsection{FITS}

% Authors: Erik B.


\subsubsection{ASCII table formats}

% Authors: Tom A.


\subsubsection{Virtual Observatory tables}

% Authors: Mike D.


\subsection{World Coordinate Systems}

% Authors: Mike D.


\subsection{Cosmology}

% Authors: Neil Crighton, Alex Conley


\section{Development workflow}

\label{sec:workflow}

% Describe tools used in the development process, and how development is managed (not specifically how we use git, but higher-level.

% Authors: Tom R., Erik T., Perry G.


\section{Planned functionality}

\label{sec:future}

% Authors: Tom R., Erik T., Perry G.


\section{Summary}

\label{sec:summary}

% Authors: Tom R., Erik T., Perry G.

\bibliographystyle{apj_custom}
\bibliography{apj-jour,references}

\end{document}
