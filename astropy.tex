\documentclass[traditabstract]{aa}

\usepackage{url}

\usepackage{xspace}
\usepackage{natbib}
\bibpunct{(}{)}{;}{a}{}{,}

\newcommand{\astropy}{\texttt{astropy}\xspace}

\begin{document}

\titlerunning{Astropy}
\authorrunning{The Astropy Collaboration}

\title{Astropy: A Community Python Package for Astronomy}

% The author list is not final, and the order of authors may still change.
% Additional authors may be added based on non-commit-based contributions to
% the project.

% Please wrap text to 78 characters.

% I have indicated authors who have acknowledged authorsip by '%confirmed' next
% to their name.

\author{
The Astropy Collaboration
  \and
% Coordination committee
Erik Tollerud\inst{\ref{inst:irvine}}  % confirmed
  \and
Thomas P. Robitaille\inst{\ref{inst:mpia}}  % confirmed
  \and
Perry Greenfield\inst{\ref{inst:stsci}}  % confirmed
  \and
% Developers (other than coordination committee).
Michael Droettboom\inst{\ref{inst:stsci}}  % confirmed
  \and
Tom Aldcroft\inst{\ref{inst:cfa}}  % confirmed
  \and
Erik Bray\inst{\ref{inst:stsci}}
  \and
Matt Davis\inst{\ref{inst:stsci}}  % confirmed
  \and
Adam Ginsburg\inst{\ref{inst:colorado}}  % confirmed
  \and
Adrian M. Price-Whelan\inst{\ref{inst:columbia}}  % confirmed
  \and
Wolfgang Kerzendorf\inst{\ref{inst:toronto}}
  \and
Alexander Conley\inst{\ref{inst:colorado}}
  \and
Neil Crighton\inst{\ref{inst:mpia}}  % confirmed
  \and
Kyle Barbary\inst{\ref{inst:argonne}}  % confirmed
  \and
Demitri Muna\inst{\ref{inst:nyu}}  % confirmed
  \and
Henry Ferguson\inst{\ref{inst:stsci}}
  \and
Frederic Grollier
  \and
Prasanth H. Nair\inst{\ref{inst:freelance}}  % confirmed
  \and
Hans M. G\"unther\inst{\ref{inst:cfa}}  % confirmed
  \and
Christoph Deil\inst{\ref{inst:mpik}}  % confirmed
  \and
Simon Conseil\inst{\ref{inst:oamp}}
  \and
Julien Woillez
  \and
Roban Kramer  % confirmed
  \and
James Turner\inst{\ref{inst:gemini_s}}  % confirmed
  \and
Leo Singer\inst{\ref{inst:ligo}}  % confirmed
  \and
% Other contributors, alphabetical, including coordination meeting
% participants
K. Azalee Bostroem\inst{\ref{inst:stsci}}  % confirmed
  \and
Doug Burke\inst{\ref{inst:cfa}}  % confirmed
  \and
Andy Casey\inst{\ref{inst:stromlo}}  % confirmed
  \and
Steve Crawford\inst{\ref{inst:saao}}
  \and
Nadia Dencheva\inst{\ref{inst:stsci}}  % confirmed
  \and
Justin Ely\inst{\ref{inst:stsci}}  % confirmed
  \and
Tim Jenness\inst{\ref{inst:jac}}  % confirmed
  \and
Kathleen Labrie\inst{\ref{inst:gemini_n}}  % confirmed
  \and
Pey Lian Lim\inst{\ref{inst:stsci}}  % confirmed
  \and
Francesco Pierfederici\inst{\ref{inst:stsci}}  % confirmed
  \and
Andrew Pontzen\inst{\ref{inst:oxford}}  % confirmed
  \and
Andy Ptak\inst{\ref{inst:gsfc}}  % confirmed
  \and
Brian Refsdal  % confirmed
  \and
Mathieu Servillat\inst{\ref{inst:saclay}}  % confirmed
  \and
Ole Streicher\inst{\ref{inst:leibniz}}  % confirmed
}

\institute{
  Center for Cosmology, Department of Physics and Astronomy, University of California at Irvine, Irvine, CA, 92697, USA
  \label{inst:irvine}
    \and
  Max Planck Institute for Astronomy, K\"onigstuhl 17, Heidelberg 69117, Germany
  \label{inst:mpia}
    \and
  Space Telescope Science Institute, 3700 San Martin Drive, Baltimore, MD 21218
  \label{inst:stsci}
    \and
  Harvard-Smithsonian Center for Astrophysics, 60 Garden Street, Cambridge, MA, 02138, USA
  \label{inst:cfa}
    \and
  Center for Astrophysics and Space Astronomy, University of Colorado, Boulder, CO 80309, USA
  \label{inst:colorado}
    \and
  Department of Astronomy, Columbia University, Pupin Hall, 550W 120th St., New York, NY 10027, USA
  \label{inst:columbia}
    \and
  Department of Astronomy and Astrophysics, University of Toronto, 50 Saint George Street, Toronto, ON M5S3H4, Canada
  \label{inst:toronto}
    \and
  Argonne National Laboratory, High Energy Physics Division, 9700 South Cass Avenue, Argonne, Illinois 60439, USA
  \label{inst:argonne}
    \and
  Center for Cosmology and Particle Physics, New York University, New York, NY 10003, USA
  \label{inst:nyu}
    \and
  Freelance
  \label{inst:freelance}
    \and
  Max-Planck-Institute for Nuclear Physics, P.O. Box 103980, 69029 Heidelberg, Germany
  \label{inst:mpik}
    \and
  Laboratoire d'Astrophysique de Marseille, OAMP, Universit\'e Aix-Marseille et CNRS,
Marseille, France
  \label{inst:oamp}
    \and
  Gemini Observatory, Casilla 603, La Serena, Chile
  \label{inst:gemini_s}
    \and
  LIGO Laboratory, California Institute of Technology, MC 100-36, 1200 E. California Blvd., Pasadena, CA, 91125, USA
  \label{inst:ligo}
    \and
  Research School of Astronomy and Astrophysics, Australian National University, Mount Stromlo Observatory
  \label{inst:stromlo}
    \and
  SAAO, P.O. Box 9, Observatory 7935, Cape Town, South Africa
  \label{inst:saao}
    \and
  Joint Astronomy Centre, 660 North A'ohoku Place, Hilo, HI 96720, USA
  \label{inst:jac}
    \and
  Gemini Observatory, 670 N. A'ohoku Place, Hilo, Hawaii 96720, USA
  \label{inst:gemini_n}
    \and
  Oxford Astrophysics, Denys Wilkinson Building, Keble Road, Oxford OX1 3RH, UK; Balliol College, Broad Street, Oxford OX1 3BJ, UK
  \label{inst:oxford}
    \and
  NASA Goddard Space Flight Center, X-ray Astrophysics Lab Code 662, Greenbelt, MD 20771, USA
  \label{inst:gsfc}
    \and
  Laboratoire AIM (CEA/DSM/IRFU/SAp, CNRS, Universite Paris Diderot), CEA Saclay, Bat. 709, 91191 Gif-sur-Yvette, France
  \label{inst:saclay}
    \and
  Leibniz Institute for Astrophysics Potsdam (AIP)
  \label{inst:leibniz}
}

\abstract{
We present the first public version of the open-source and community-developed
Python package, Astropy. This package aims to provide core astronomy-related
functionality to the community, including for example support for
domain-specific file formats such as Flexible Image Transport System (FITS)
files, Virtual Observatory (VO), and common ASCII table formats, unit and
physical quantity conversions, physical constants specific to astronomy,
celestial coordinate and time transformations, world coordinate system (WCS)
support, generalized containers for representing gridded as well as tabular
data, and a framework for cosmological transformations and conversions.
Significant functionality is under active development, such as a model fitting
framework, VO client and server tools, and aperture and point spread function
(PSF) photometry tools. Participation in development is open to anyone
interested in joining the project.

}

\maketitle

\tableofcontents


\section{Introduction}

% Authors: Tom R., Erik T., Perry G.

The Python programming language\footnote{\url{http://www.astropy.org}} has been
one of the fastest-growing programming languages in the astronomy community in
the last decade. While there have been a number of efforts to develop Python
packages for astronomy-specific functionality, these efforts have been
fragmented, and several dozens of packages have been developed across the
community with little or no coordination, leading to duplication and a lack of
homogeneity across packages. This in turn has made it difficult for users to
get set up with all the required packages needed in an Astronomer's toolkit.
Since a number of these packages depend on individual or small groups of
developers, packages are sometimes no longer maintained, or simply become
unavailable, which is detrimental to long-term research and research
reproducibility.

The Astropy project was started\footnote{Following discussions on
the \texttt{astropy} mailing list at
\url{http://mail.scipy.org/mailman/listinfo/astropy}} in 2011 out of a desire
to bring together developers across the field of astronomy in order to
coordinate efforts to develop a common Python library. The aim of this library
was to cover much of the astronomy-specific functionality needed by
researchers, complementing more general standard scientific packages such as
NumPy \citep{oliphant2006guide,van2011numpy} and SciPy \citep{jones2001scipy},
which are invaluable for numerical array-based calculations, and more general
scientific algorithms (e.g. interpolation, integration, clustering, etc.)
respectively. To date, over 130 people are signed up to the
\textit{development} mailing list for the Astropy project\footnote{
\url{https://groups.google.com/forum/?fromgroups\#!forum/astropy-dev}}.

Most efforts in the Astropy project to date have gone towards developing the
core \astropy package. However, we note the scope of the Astropy project is
not simply to create a core package, but more generally to bring together
authors of existing packages to work together and agree on including a common
set of functionality in the core package (and in some cases deprecating their
packages if all functionality ends up in \astropy), since the aim is
ultimately to simplify the landscape of packages available to users. In
addition, the Astropy project includes more specialized Python packages (which
we call \textit{affiliated} packages) that are not included in the core
package for various reasons: for some the functionality is in early stages of
development and is not robust; or the license is not compatible with Astropy;
the package includes large files; or the functionality is mature, but too
specific to be included in a core package.


In this paper, we present the first public release (v0.2) of the \astropy
package. We provide an overview of the current capabilities
(\S\ref{sec:capabilities}), our development workflow (\S\ref{sec:workflow}),
and planned functionality (\S\ref{sec:future}).


\section{Capabilities}

\label{sec:capabilities}


\subsection{Units, Quantities, and Physical Constants}

% Authors: Perry, Mike, Adrian, Tom R.

The \texttt{astropy.units} package provides support for physical
units.  It is based on the code in the \texttt{pynbody} package
written by Andrew Pontzen, but has since diverged considerably in
behavior and implementation.

Converting between units is simple:

\begin{verbatim}
>>> from astropy import units as u
>>> # Convert from parsec to meter
>>> u.pc.to(u.m)
3.0856776e+16
\end{verbatim}

The user can also define their own units, either as standalone base
units or by composing other units together.  This is done using simple
mathematical operators.  For example:

\begin{verbatim}
>>> # Define some custom units
>>> cms = u.cm / u.s
>>> mph = u.mile / u.hour
>>> cms.to(mph, 1)
0.02236936292054402
\end{verbatim}

Astropy knows about certain physical ``concepts'' and is able to
deduce the purpose of a unit, even if it is user-defined.  Using the
example of mile-per-hour (\texttt{mph}) from above:

\begin{verbatim}
>>> mph = u.mile / u.hour
>>> mph.physical_type
u'speed'
\end{verbatim}

\texttt{astropy.units} includes unit definitions in both the
International System of Units (SI) and the Centimeter-Gram-Second
(CGS) systems, as well as a number of astronomy- and
astrophysics-specific units.  Converting between systems is possible:

\begin{verbatim}
>>> u.Pa.to_system(u.cgs)
[Unit("1.000000e+01 Ba")]
\end{verbatim}

It's also possible to decompose units into their base units, and then
pattern match those parts against known units.

\begin{verbatim}
>>> u.Hz.decompose()
Unit("1 / (s)")
>>> (1.0 / u.s).compose()
>>> (u.s ** -1).compose()
[Unit("Hz"), ...]
\end{verbatim}

Also included is the concept of ``equivalencies''.  These are mappings
between units that, while not strictly convertible, can be considered
convertible under certain conditions.  For example, we include an
equivalency mapping between frequency and wavelength in a vacuum.  A
conversion from wavelength to frequency is normally rejected:

\begin{verbatim}
>>> u.nm.to(u.Hz, [1000, 2000])
UnitsException: 'nm' (length) and 'Hz'
(frequency) are not convertible
\end{verbatim}

However, by passing an equivalency list (\texttt{u.spectral()})}, it
does:

\begin{verbatim}
>>> u.nm.to(u.Hz, [1000, 2000],
            equivalencies=u.spectral())
array([  2.99792458e+14,   1.49896229e+14])
\end{verbatim}

It's well known that there are multiple string representations for
units used in the astronomy community.  The FITS Standard
\cite{fits2008} defines a unit standard, as well as both the Centre de
Donn\'ees astronomiques de Strasbourg (CDS) \citep{ochsenbein2000cds}
and the NASA/Goddard's Office of Guest Investigator Programs (OGIP)
\citep{george1995ogip}.  In additional, the International Virtual
Observatory Alliance (IVOA) has a forthcoming VOUnit standard
\citep{derriere2012vounit} in an attempt to resolve some of these
differences.  Rather than decide, \texttt{astropy.units} supports all
of these standards\footnote{OGIP support is forthcoming at the time of
  this writing.}, and allows the user to select the appropriate one
when reading and writing unit string definitions to and from external
file formats.

Also included in the \texttt{astropy.units} package is the
\texttt{Quantity} object, which represents a numerical value with an
associated unit.  This objects support arithmetic with other numbers
and \texttt{Quantity} objects and preserve their units:

\begin{verbatim}
>>> x = 15.1 * u.m / (32.0 * u.s)
>>> x
<Quantity 0.471875 m / (s)>
>>> x.unit
Unit("m / (s)")
>>> x.value
0.471875
\end{verbatim}

These \texttt{Quantity} objects are used to define a number of useful
astronomical constants included with \texttt{astropy}, each with an
associated unit (where applicable) and additional metadata describing
their provenance and uncertainties.

\begin{verbatim}
>>> from astropy.constants import si
>>> print si.c
  Name   = Speed of light in vacuum
  Value  = 299792458.0
  Error  = 0.0
  Units = m / (s)
  Reference = CODATA 2010
>>> print si.c.to('km/s')
299792.458 km / (s)
\end{verbatim}

\subsection{Time}

% Authors: Tom A.


\subsection{Celestial Coordinates}

% Authors: Erik T., Adrian, Demitri


\subsection{Tables and Gridded data}

% Authors: Tom A., Tom R., Wolfgang K., Erik T.

\label{sec:table}

The \texttt{astropy.table} and \texttt{astropy.nddata} sub-packages contain
classes (\texttt{Table} and \texttt{NDData}) that allow users to represent
astronomical data in the form of tables or n-dimensional gridded datasets,
including meta-data (for example units).

The \texttt{Table} class provides a high-level wrapper to Numpy structured
arrays, which are essentially arrays that have fields (or columns) with
heterogeneous data types, and any number of rows. Numpy structured arrays are
however difficult to manipulate or modify, so the \texttt{Table} class makes
it easy for users to create a table from columns, add/remove columns or rows,
and mask values from the table. In addition, tables can be easily
read/written from/to common file formats using the \texttt{Table.read} and
\texttt{Table.write} methods. In addition to providing easy manipulation and
input/output of table objects, the \texttt{Table} class allows units to be
specified for each column using the \texttt{astropy.units} framework, and
also allows the \texttt{Table} object to contain arbitrary meta-data (stored
in \texttt{Table.meta}).


The \texttt{NDData} class ...

\subsection{File Formats}


\subsubsection{FITS}

% Authors: Erik B.


\subsubsection{ASCII table formats}

% Authors: Tom A.


\subsubsection{Virtual Observatory tables}

% Authors: Mike D.
\texttt{astropy.io.votable} (formerly the standalone project
\texttt{vo.table}) provides full support for reading and writing
VOTable format files versions 1.1 and 1.2
\citep{ochsenbein2004votable,ochsenbein2009votable}.  It efficently
stores the tables in memory as Numpy structured arrays.  The file is
read using streaming to avoid reading in the entire file at once and
greatly reducing the memory footprint.

It is possible to convert any one of the tables in a VOTable file to
an \texttt{astropy.table.Table} object (\S\ref{sec:table}), where it
can be edited and then written back to a VOTable file without any loss
of data.

The VOTable standard is not strictly adhered to by all VOTable file
writers in the wild.  Therefore, \texttt{astropy.io.votable} provides
a number of tricks and workarounds to support as many VOTable sources
as possible, whenever the result would not be ambiguous.  A validation
tool (\texttt{volint}) is also provided that outputs recommendations
to improve the standard compliance of a given file, as well as
validate it against the official VOTable schema.

Support for VOTable 1.3 is planned for the future.

\subsection{World Coordinate Systems}

% Authors: Mike D.

\texttt{astropy.wcs} contains utilities for managing World Coordinate
System (WCS) transformations in FITS files.  These transformations map
the pixel locations in an image to their real-world units, such as
their position on the celestial sphere.  This library is specific to
WCS as it relates to FITS as described in the FITS WCS papers
\citep{greisen2002wcs,calabretta2002wcs,greisen2006wcs} and is
distinct from a planned Astropy package that will handle WCS
transformations in general, regardless of their representation.

\texttt{astropy.wcs} is a wrapper around Mark Calabretta's
\texttt{wcslib} \citep{calabretta2013wcslib}.  Since all of the FITS
header parsing is done using \texttt{wcslib}, it is assured the same
behavior as the many other tools that use \texttt{wcslib}.  On top of
the basic FITS WCS support, it adds support for the Simple Imaging
Polynomial (SIP) convention and table lookup distortions as defined in
the draft WCS ``Paper IV'' \citep{calabretta2004wcs}.  Each of these
transformations can be used independently or together in a fixed
pipeline.

\texttt{astropy.wcs} also serves as a useful FITS WCS validation tool,
as it is able to report on many common mistakes or deviations from the
standard in a given FITS file.

\subsection{Cosmology}

% Authors: Neil Crighton, Alex Conley


\section{Development workflow}

\label{sec:workflow}

% Describe tools used in the development process, and how development is managed (not specifically how we use git, but higher-level.

% Authors: Tom R., Erik T., Perry G.


\section{Planned functionality}

\label{sec:future}

% Authors: Tom R., Erik T., Perry G.


\section{Summary}

\label{sec:summary}

% Authors: Tom R., Erik T., Perry G.

\bibliographystyle{apj_custom}
\bibliography{apj-jour,references}

\end{document}
